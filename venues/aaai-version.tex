% AAAI Version - 7 pages (AI Safety Angle)
% SigmaGov: Formal Specifications for Verifiable LLM Agent Alignment

\documentclass[letterpaper]{article}
\usepackage{aaai25}  % AAAI 2025 style
\usepackage{times}
\usepackage{helvet}
\usepackage{courier}
\usepackage{amsmath,amssymb}
\usepackage{booktabs}
\usepackage{listings}
\usepackage{xcolor}

\newcommand{\sigmagov}{\ensuremath{\Sigma\text{Gov}}}
\newcommand{\oblig}{\ensuremath{\mathbf{O}}}
\newcommand{\forb}{\ensuremath{\mathbf{F}}}

\lstset{basicstyle=\ttfamily\footnotesize,breaklines=true,frame=single}

\title{SigmaGov: Formal Specifications for Verifiable LLM Agent Alignment}
\author{Rui Wang\\University of Houston\\rwang19@uh.edu}

\begin{document}
\maketitle

\begin{abstract}
As LLM agents operate with increasing autonomy, alignment approaches based on
RLHF and constitutional AI lack formal guarantees---we cannot \emph{prove}
correct behavior, only observe it. We present \sigmagov{}, a formal governance
calculus with Lean 4 formalization providing: (1) eight foundational axioms for
agent governance, (2) binary deontic logic eliminating behavioral ambiguity,
(3) semantic convergence checking for purpose verification, and (4) workflow
algebra with purpose preservation proofs. Our complete formalization (15 files,
~5,500 lines, zero incomplete proofs) demonstrates that formal methods can
provide the missing foundation for verifiably aligned LLM agent systems.
\end{abstract}


\section{Introduction}

The deployment of increasingly autonomous LLM agents~\cite{wang2024survey}
raises critical safety questions. Current alignment approaches---RLHF
\cite{ouyang2022training}, constitutional AI~\cite{bai2022constitutional}---provide
statistical guarantees but cannot \emph{prove} behavioral correctness.

We argue that \textbf{formal specifications} are necessary for verifiable
alignment. Formal methods have secured critical systems from compilers
\cite{leroy2009compcert} to operating systems~\cite{klein2009sel4}. Why not
LLM agents?

We present \sigmagov{}, a formal governance calculus with machine-verified
specifications in Lean 4. Our key insight: agent governance can be formalized
as deontic logic~\cite{vonwright1951} over behavioral propositions.

\subsection{The Alignment Gap}

Consider an agent tasked with ``implement authentication.'' Without formal
specifications:
\begin{itemize}
\item How do we \emph{verify} task completion?
\item What constraints must be respected?
\item How does purpose propagate through delegation?
\end{itemize}
\sigmagov{} answers these with machine-checkable specifications.


\section{SigmaGov Framework}

\subsection{Binary Governance}

A key design decision is eliminating behavioral ambiguity:

\textbf{Axiom T5 (Binary Governance):} $\forall \phi: \oblig(\phi) \oplus \forb(\phi)$

Every behavior is either obligatory or forbidden---no ``permitted but not
required'' middle ground. An agent always knows whether an action should occur.

\subsection{Core Axioms for AI Safety}

\textbf{T0 (Truthfulness):} All outputs must be grounded or acknowledge uncertainty.
This prevents fabrication and hallucination at the specification level.

\textbf{T1 (Purpose Seeding):} Every prompt seeds exactly one purpose.
Agents cannot invent purposes beyond user intent.

\textbf{T6 (Achievement Dimensions):} Completion requires four dimensions:
WHAT (40\%), WHERE (20\%), HOW (25\%), WHY (15\%). Agents cannot claim
completion without verifying all dimensions.

\subsection{Semantic Convergence}

NPL (Natural Procedural Language) measures alignment between stated purpose
and reported result using embedding similarity ($\tau = 0.85$). This provides
a formal criterion for ``task complete'' rather than agent self-report.


\section{Formalization Results}

\begin{table}[h]
\centering
\small
\begin{tabular}{lrl}
\toprule
\textbf{Component} & \textbf{Lines} & \textbf{Status} \\
\midrule
Foundational Axioms (T0--T8) & 495 & Complete \\
Five-Pillar Architecture & 288 & Complete \\
NPL Convergence & 549 & Complete \\
Workflow Algebra & 349 & Complete \\
Test Correspondence & 757 & Complete \\
Supporting Modules & 3,062 & Complete \\
\midrule
\textbf{Total} & \textbf{5,500} & \textbf{Zero sorry} \\
\bottomrule
\end{tabular}
\caption{Lean 4 Formalization Statistics}
\end{table}

The formalization compiles with zero \texttt{sorry} declarations (incomplete
proofs), demonstrating internal consistency.

\section{Related Work}

\textbf{Neural Network Verification.} Reluplex~\cite{katz2017reluplex} verifies
input-output properties of DNNs. \sigmagov{} operates at the agent behavioral level.

\textbf{AI Alignment.} RLHF~\cite{christiano2017deep} and Constitutional AI
\cite{bai2022constitutional} provide training-time alignment. We complement
these with formal runtime specifications.

\textbf{Normative MAS.} Multi-agent norms~\cite{boella2006introduction} inform
our approach to behavioral constraints across agent hierarchies.


\section{Conclusion}

We presented \sigmagov{}, demonstrating that formal methods can provide
verifiable foundations for LLM agent alignment. Our Lean 4 formalization
(8 axioms, 5-pillar architecture, NPL convergence, workflow algebra)
comprises 5,500 lines with zero incomplete proofs.

\textbf{Impact for AI Safety:} \sigmagov{} enables:
\begin{itemize}
\item \emph{Provable behavioral constraints} via deontic logic
\item \emph{Verifiable task completion} via NPL convergence
\item \emph{Purpose preservation} across agent hierarchies
\end{itemize}

Future work includes runtime verification integration and multi-agent
coordination specifications.

\section*{Reproducibility}
Lean 4 source available in supplementary materials.

\bibliographystyle{aaai}
\bibliography{references}

\end{document}
