% arXiv Version - Full Paper (No page limit)
% SigmaGov: A Formal Governance Calculus for LLM Agent Systems

\documentclass[11pt]{article}
\usepackage[utf8]{inputenc}
\usepackage[T1]{fontenc}
\usepackage{amsmath,amssymb,amsthm}
\usepackage{mathtools}
\usepackage{listings}
\usepackage{xcolor}
\usepackage{hyperref}
\usepackage{booktabs}
\usepackage[margin=1in]{geometry}

% arXiv identifier (fill after submission)
% \arxivid{26XX.XXXXX}

% Theorem environments
\newtheorem{theorem}{Theorem}[section]
\newtheorem{axiom}{Axiom}
\newtheorem{definition}[theorem]{Definition}

% Custom commands
\newcommand{\sigmagov}{\ensuremath{\Sigma\text{Gov}}}
\newcommand{\oblig}{\ensuremath{\mathbf{O}}}
\newcommand{\forb}{\ensuremath{\mathbf{F}}}


% Lean 4 code styling
\lstdefinelanguage{Lean}{
  keywords={theorem,lemma,def,axiom,structure,inductive,where,forall,exists},
  keywordstyle=\color{blue}\bfseries,
  commentstyle=\color{green!50!black}\itshape,
  morecomment=[l]{--},
  morecomment=[s]{/-}{-/},
}
\lstset{language=Lean,basicstyle=\ttfamily\small,breaklines=true,frame=single}

\title{\textbf{SigmaGov: A Formal Governance Calculus for LLM Agent Systems}\\
\large{Machine-Verified Specifications in Lean 4}}

\author{
  Rui Wang\\
  University of Houston\\
  \texttt{rwang19@uh.edu}
}

\date{January 2026}

\begin{document}
\maketitle

\begin{abstract}
As Large Language Model (LLM) agents become increasingly autonomous, ensuring
behavior aligns with specified intentions becomes critical. We present \sigmagov{},
a formal governance calculus providing machine-checkable specifications for LLM
agent behavior, purpose tracking, and achievement verification.

Our Lean 4 formalization establishes: (1) eight foundational axioms (T0--T8)
governing truthfulness, purpose immutability, and achievement dimensions;
(2) a five-pillar architecture decomposing agent execution into semantic,
syntactic, autonomous, memory, and contextual domains; (3) Natural Procedural
Language (NPL) for measuring semantic convergence; and (4) a workflow algebra
with compositional operators preserving purpose across layer hierarchies.

The complete formalization comprises 14 Lean 4 source files (~5,300 lines) with
\textbf{zero incomplete proofs}. \sigmagov{} provides a rigorous foundation
for building verifiably aligned LLM agent systems.
\end{abstract}


\section{Introduction}

The rapid advancement of LLM capabilities has enabled increasingly autonomous
agent systems~\cite{wang2024survey}. These agents perform complex multi-step
tasks, delegate to sub-agents, and make consequential decisions with minimal
human oversight. This autonomy raises fundamental questions about \emph{governance}:
How do we specify intended behavior? How do we verify purpose achievement?

Existing approaches to LLM alignment focus on training-time interventions
(RLHF~\cite{ouyang2022training}, constitutional AI~\cite{bai2022constitutional})
or runtime guardrails. While valuable, these lack \emph{formal guarantees}.

We present \sigmagov{}, a formal governance calculus addressing these limitations
through machine-verified specifications. Our key insight: LLM agent governance
can be formalized as a \emph{deontic logic}~\cite{vonwright1951} over behavioral
propositions, combined with \emph{semantic convergence checking}.

\subsection{Contributions}
\begin{enumerate}
    \item \textbf{Foundational Axiom System (T0--T8)}: Eight axioms governing
    truthfulness, purpose immutability, binary achievement, and four-dimensional
    achievement verification.
    \item \textbf{Five-Pillar Architecture}: Decomposition into $\Phi_{\text{sem}}$,
    $\Phi_{\text{syn}}$, $\Phi_{\text{auto}}$, $\Phi_{\text{mem}}$, $\Phi_{\text{ctx}}$.
    \item \textbf{NPL Convergence}: Semantic alignment measurement with formal
    thresholds ($\tau_{\text{npl}} = 0.85$).
    \item \textbf{Workflow Algebra}: Compositional operators ($\bind$, $\seq$,
    $\parallel$) with purpose preservation proofs.
    \item \textbf{Complete Lean 4 Formalization}: 14 source files, ~5,300 lines, zero sorry.
\end{enumerate}


\section{The \sigmagov{} Formalization}

\subsection{Deontic Operators}

A key design decision is \textbf{binary governance}:

\begin{axiom}[T5: Binary Governance]
$\forall \phi: \oblig(\phi) \oplus \forb(\phi)$
\end{axiom}

Every behavior is either obligatory or forbidden---no ``permitted but not
required'' middle ground. This eliminates ambiguity.

\subsection{Foundational Axioms (T0--T8)}

\begin{axiom}[T0: Truthfulness]
$\forall o: \text{valid}(o) \Leftrightarrow \text{grounded}(o) \lor \text{ack\_uncertainty}(o)$
\end{axiom}

\begin{axiom}[T1: Purpose Seeding]
$\forall p \in \text{UserPrompt}: \exists! P: P = \text{purpose}(p)$
\end{axiom}

\begin{axiom}[T6: Achievement Dimensions]
$\text{Achievement} \equiv \text{WHAT} \land \text{WHERE} \land \text{HOW} \land \text{WHY}$

With weights: $\gamma_{\text{what}}=0.40$, $\gamma_{\text{where}}=0.20$,
$\gamma_{\text{how}}=0.25$, $\gamma_{\text{why}}=0.15$.
\end{axiom}

\begin{axiom}[T7: Layer Self-Containment]
$\forall L_n: \text{config}(L_n).\text{complete} = \text{true}$
\end{axiom}

\begin{axiom}[T8: Context Anchoring]
$\forall E: \exists M = (\text{cwd}, t): \text{anchored}(E, M) \land \text{immutable}(M)$
\end{axiom}


\subsection{Five-Pillar Architecture}

\begin{table}[h]
\centering
\begin{tabular}{lll}
\toprule
\textbf{Pillar} & \textbf{Domain} & \textbf{Responsibility} \\
\midrule
$\Phi_{\text{sem}}$ & Semantic & Reasoning, interpretation \\
$\Phi_{\text{syn}}$ & Syntactic & Formal rules, grammar \\
$\Phi_{\text{auto}}$ & Autonomous & Agency, execution \\
$\Phi_{\text{mem}}$ & Memory & State persistence \\
$\Phi_{\text{ctx}}$ & Context & Spacetime anchoring \\
\bottomrule
\end{tabular}
\caption{Five-Pillar Architecture}
\end{table}

\subsection{NPL Convergence}

NPL provides the semantic bridge between purpose and result:

\begin{definition}[NPL Convergence]
$\text{nplConverged}(P, R, \tau, \epsilon) \Leftrightarrow 
  \text{nplSimilar}(P, R, \tau) \land \text{nplStructurallyAligned}(P, R, \epsilon)$
\end{definition}

Where $\tau_{\text{npl}} = 0.85$ and similarity uses cosine distance on embeddings.


\subsection{Workflow Algebra}

Workflows are modeled as a Reader monad over Purpose:

\begin{lstlisting}[caption={Workflow Operators}]
def bind (w : Workflow a) (f : a -> Workflow b) : Workflow b :=
  fun p => w p >>= fun a => f a p

def seq (w1 : Workflow a) (w2 : Workflow b) : Workflow b :=
  fun p => w1 p >> w2 p

def par (w1 : Workflow a) (w2 : Workflow b) : Workflow (a * b) :=
  fun p => do let r1 <- w1 p; let r2 <- w2 p; pure (r1, r2)
\end{lstlisting}

\begin{theorem}[Purpose Preservation]
Sequential composition preserves purpose:
$\forall w_1, w_2, p: (w_1 \seq w_2)(p) = w_1(p) \bind (\lambda\_. w_2(p))$
\end{theorem}

\section{Evaluation}

\begin{table}[h]
\centering
\begin{tabular}{lrrl}
\toprule
\textbf{File} & \textbf{Lines} & \textbf{Theorems} & \textbf{Status} \\
\midrule
Basic.lean & 295 & 5 & Complete \\
Axioms.lean & 495 & 12 & Complete \\
NPL.lean & 549 & 8 & Complete \\
Workflow.lean & 349 & 6 & Complete \\
TestCorrespondence.lean & 757 & 38 & Complete \\
\emph{+ 9 others} & 3,055 & 31 & Complete \\
\midrule
\textbf{Total} & \textbf{5,300} & \textbf{100} & \textbf{Zero sorry} \\
\bottomrule
\end{tabular}
\caption{Formalization Statistics}
\end{table}


\section{Related Work}

\textbf{Formal Methods for AI Safety.} Prior work on neural network
verification~\cite{katz2017reluplex,huang2017safety} focuses on input-output
properties. \sigmagov{} formalizes behavioral governance at the agent level.

\textbf{Deontic Logic.} Classical deontic logic~\cite{vonwright1951,mcnamara2010}
provides the foundation for our obligation/forbiddance operators.

\textbf{LLM Agent Frameworks.} Systems like ReAct~\cite{yao2023react},
LangChain~\cite{chase2022langchain}, and Claude~\cite{anthropic2024claude}
provide practical tooling but lack formal specifications.

\textbf{Normative Multi-Agent Systems.} Work on normative MAS~\cite{boella2006introduction}
informs our approach to behavioral constraints across agent hierarchies.

\section{Conclusion}

We presented \sigmagov{}, a formal governance calculus for LLM agent systems
with complete Lean 4 formalization. Key contributions: eight foundational
axioms (T0--T8), five-pillar architecture, NPL convergence, workflow algebra
with purpose preservation, and zero-sorry formalization (~5,300 lines).

Future work includes multi-agent coordination, runtime verification integration,
and certified code extraction from Lean specifications.

\section*{Code Availability}
Full Lean 4 source: \url{https://github.com/fixthatbug/sigmagov-lean4}

\bibliographystyle{plain}
\bibliography{references}

\end{document}
